\documentclass[a4paper, 11pt]{beamer}
\usepackage{graphicx}
\usepackage{listings}
\usepackage{mathtools}
\usepackage[utf8]{inputenc}
\usepackage{polski}
%\usepackage{antiqua}
\usepackage{color}
\usepackage[T1]{fontenc}
\usetheme{Berkeley} %motyw
%Deklaracja kolorów
\begin{document}
%
\title{Podsumowanie 1 sprintu - symulacja ruchu pociagów}
\author{Prezentację tworzył Adam Mościcki}
\institute{we współpracy z Michał Pluta, Jan Wiśniewski, Edwiń Jarośinski, Paweł Kowalczyk , Margarita Chirillova}
%
\begin{frame} %ramka, nie slajd!!!
\titlepage %strona tytulowa
\end{frame}

\begin{frame}
\tableofcontents
\end{frame}
\section{Sukcesy i porażki realizacji sprintu 1}
\subsection{Zarys dokonań}
\begin{itemize}
\frametitle{Zarys dokonań}
\begin{frame}
\item symulacja wizualizowana prostymi elementami graficznymi 
\item punkty (sprzyżowania, stacje) połączone liniami prostymi (torami) oraz linie innego koloru (pociągi)
\item pociągi nie są podzielone na wagony, ale mogą przyjmować dowolną długość daną przez użytkownika 
\item część graficzna aplikacji została wykonana przy pomocy QT OpenGL 
\item mapę opisuje plik XML 
\item pociągi jadą wyznaczoną trasą zatrzymując się póki co tylko na skrzyżowaniach w przypadku kiedy segment na który chcą wjechać jest zajęty przez inny pociąg - trasa jest wyznaczana algorytmem Dijkstry 
\item mapa może dowolnie duża, jest możliwość przesuwania ekranu, jeśli mapa nie mieści się na jednym ekranie
\end{frame}
\end{itemize}
\subsection{Obserwator}
\begin{enumerate}
\begin{frame}
\frametitle{Obserwator}
\item Jako obserwator chciałbym :
\begin{itemize}
\item {\color{green}b) widzieć jak poruszają się pociągi (must have)
\item d) mieć możliwość zatrzymania symulacji w dowolnej chwili (must have)
\item g) aby stacje były reprezentowane na mapie w postaci graficznej (must have)
\item k) aby program miał możliwość zapisania do pliku przebiegu symulacji (must have)}
\end{itemize}
\end{frame}
\subsection{Projektant}
\begin{frame}
\frametitle{Projektant}
\item Jako projektant torów chciałbym :
\begin{itemize}
\item {\color{green}a) mieć możliwość ustawiania stacji w dowolnym punkcie mapy (must have)
\item b) mieć możliwość łączenia stacji torami (jednokierunkowymi lub dwukierunkowymi) (must have)
\item e) projektować skrzyżowania torów (must have)}
\item {\color{orange}f) ustawiać priotytety przejazdów pociągów na poszczególych skrzyżowaniach (should have)}
\item {\color{green}g) mieć możliwość łaczenia wielu odcinków torów i wiele skrzyżowań w segment. Na jednym segmencie nie może znajdować się więcej niż jeden pociąg (nice to have)
\item h) aby plik opisujący mapę miał składnię umożliwiającą względnie nieskomplikowaną edycję mapy (must have)}
\end{itemize}
\end{frame}
\subsection{Logistyk}
\begin{frame}
\frametitle{Logistyk}
\item Jako logistyk chciałbym
\begin{itemize}
\item {\color{orange}b) edytując plik wejściowy ustawiać ilość wagonów przed symulacją (should have)}
\item {\color{green}h) aby pociąg miał możliwość wyznaczenia właściwej dla siebie trasy na podstawie danych stacji: początkowej i końcowej, a także dowolnej ilości punktów (stacji) "przez" (must have)}
\end{itemize}
\end{frame}
\end{enumerate}
\section{Bonusy}
\begin{frame}
Rzeczy zrealizowane ponadwymiarowo:
\begin{itemize}
\item przyspieszanie i zwolnianie symulacji
\item możliwość ustalenia dowolnej odległości między stacjami niezaleznie od reprezentacji graficznej
\item możliwość przypisać pociągowi trasę specyfikując stacje przez które ma przejechać
\item yworzyć dowolnie dużą mapę, to chyba mamy
\end{itemize}
\end{frame}
\end{document}


