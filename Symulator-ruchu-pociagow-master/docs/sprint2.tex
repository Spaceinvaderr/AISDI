\documentclass[a4paper, 11pt]{beamer}
\usepackage{graphicx}
\usepackage{listings}
\usepackage{mathtools}
\usepackage[utf8]{inputenc}
\usepackage{polski}
%\usepackage{antiqua}
\usepackage{color}
\usepackage[T1]{fontenc}
\usetheme{Berkeley} %motyw
%Deklaracja kolorów
\begin{document}
%
\title{Podsumowanie 2 sprintu - symulacja ruchu pociagów}
\author{Prezentację tworzył Adam Mościcki}
\institute{we współpracy z Michał Pluta, Jan Wiśniewski, Edwin Jarosiński, Paweł Kowalczyk , Margarita Chirillova}
%
\begin{frame} %ramka, nie slajd!!!
\titlepage %strona tytulowa
\end{frame}

\begin{frame}
\tableofcontents
\end{frame}
\section{Sukcesy i porażki realizacji sprintu 2}
\subsection{Zarys dokonań}
\begin{itemize}
\begin{frame}
\frametitle{Zarys dokonań}
\item symulacja wizualizowana w dużo lepszy sposób niż w sprincie 1
\item kafelki z obiektami odzwierciedlają rzeczywistość
\item pociągi sa kwantyzowane wagonami oraz mogą przyjmować dowolną długość daną przez użytkownika 
\item część graficzna aplikacji została wykonana przy pomocy QT OpenGL 
\item mapę opisuje plik XML w 2 wersjach pseudograficznej i czysto xmlowej
\item tory tworzą segmenty
\item wyświetlana jest odpowiednia sygnalizacja dla pociagów
\item można cofać symulację do początku
\item można wyznaczyć odlegość między stacjami
\item mapa może dowolnie duża, jest możliwość przesuwania ekranu, jeśli mapa nie mieści się na jednym ekranie
\end{frame}
\end{itemize}
\subsection{Obserwator}
\begin{enumerate}
\begin{frame}
\frametitle{Obserwator}
\item Jako obserwator chciałbym :
\begin{itemize}
\color{green}
\item a) aby program informował o zablokowaniu się ruchu we wszystkich kierunkach na skrzyżowaniu (możliwe jest wczytanie mapy doprowadzającej do takiej sytuacji) (may have)
\item e) mieć możliwość przyspieszania i zwalniania symulacji (should have)
\item j) aby przy każdym skrzyżowaniu można było sprawdzić stan wszystkich sygnalizacji (needs to have)
\item l) pociągi powinny być sterowane w sposób zapobiegający kolizjom (must have)
\item m) aby symulacja była przedstawiona w czytelnej, ładnej oprawie graficznej (should have)
\item n) aby była możliwość odtworzenia symulacji na podstawie zapisanego pliku (must have)
\end{itemize}
\end{frame}
\subsection{Projektant}
\begin{frame}
\frametitle{Projektant}
\item Jako projektant torów chciałbym :
\begin{itemize}
\color{green}
\item c) mieć możliwość kreowania otoczenia przez bardzo uproszczone elementy, takie jak przeszkody na mapie (may have)
\item d) mieć możliwość ustalenia dowolnej odległości między stacjami (niezależnie od reprezentacji graficznej trasy) (must have)
\end{itemize}
\end{frame}
\subsection{Logistyk}
\begin{frame}
\frametitle{Logistyk}
\item Jako logistyk chciałbym
\begin{itemize}
\color{green}
\item a) znać odległości pomiędzy poszczególnymi stacjami (must have)
\item g) móc przypisać pociągowi trasę specyfikując punkty (stacje) przez które ma on przejechać (must have)
\end{itemize}
\end{frame}
\end{enumerate}
\section{Bonusy}
\begin{frame}
Rzeczy zrealizowane ponadwymiarowo:
\begin{itemize}
\item pociąg dostosowuje prędkość do pociągu jadącego przed nim
\item pociąg stopniowo zwiększa swoją prędkość
\item pociąg zwalnia podczas dojeżdżania do stacji lub skrzyżowania
\end{itemize}
\end{frame}
\end{document}


